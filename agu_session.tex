%%%%%%%%%%%%%%%%%%%%%%%%%%%%%%%%%%%%%%%%%%%%%%%%%%%%%%%%%%%%%%%%%%%%%%%%%%%%%%%
\begin{frame}
 \frametitle{AGU Fall Meeting Session Advertisement}\small
 \begin{center}
  \vskip-0.15in
  \textbf{\large Big Data in the Geosciences: \\ New Analytics Methods and Parallel Algorithms}
 \end{center}

 \medskip
 \vbox{\footnotesize\textit{Co-conveners: Jitendra~Kumar (ORNL), Robert~L.~Jacob (ANL), Forrest~M.~Hoffman (ORNL), and Miguel~D.~Mahecha (MPI-Jena)}}

% \medskip
% \textbf{Confirmed Invited Presenters:}
% \begin{itemize}
%  \item Gary Geernaert (U.S. Dept. of Energy)
%  \item Matt Hancher (Google Earth Engine)
%  \item Jeff Daily (Pacific Northwest National Laboratory)
%  \item William Hargrove (USDA Forest Service)
% \end{itemize}

 \medskip
 \vbox{\footnotesize Earth and space science data are increasingly large and
 complex--often representing high spatial/temporal/spectral resolution
 and dimensions from remote sensing or model results--making such data
 difficult to analyze, visualize, interpret, and understand by traditional
 methods.  This session focuses on application and development of new
 geoscientific data analytics approaches (statistical, data mining,
 assimilation, machine learning, etc.) and parallel algorithms and
 software employing high performance computing resources for scalable
 analysis and novel applications of traditional methods on large
 geoscience data sets.  Analysis methods that operate in-situ with
 parallel simulations to reduce output data volumes are also of interest.
 Abstracts focused on analysis, synthesis and knowledge extraction from
 large and complex Earth science data from all disciplines are invited.}

 \bigskip
 \centerline{\color{red} \textbf{Abstract submissions are due 6 August 2014, 23:59 EDT/03:59 +1 GMT}}

\end{frame}
%%%%%%%%%%%%%%%%%%%%%%%%%%%%%%%%%%%%%%%%%%%%%%%%%%%%%%%%%%%%%%%%%%%%%%%%%%%%%%%
